\documentclass[11pt]{article}
%
\usepackage[parfill]{parskip}
\usepackage{amsmath}   
\usepackage{amsthm}
\usepackage{amssymb} 
\usepackage{calc}
\usepackage{centernot}
\usepackage{ifthen}
\usepackage{graphicx}
\usepackage{enumerate}
\usepackage[mathscr]{eucal}



%MATH FORMATTING
\renewcommand\bf[1]{\ensuremath{\mathbf{#1}}}
\renewcommand\cal[1]{\ensuremath{\mathcal{#1}}}
\newcommand\bb[1]{\ensuremath{\mathbb{#1}}}
\newcommand\call{\stackrel{\text{\tiny call}}{=}}
\newcommand\thru[2]{\ensuremath{#1,\ldots,#2}}
\newcommand\rank[1]{\ensuremath{\text{Rank}(#1)}}
\newcommand\tboxed[1]{\boxed{\text{#1}}}
%\renewcommand\and{\ensuremath{\;\;\;\;\;\;\text{and}\;\;\;\;\;\;} }
\newcommand\where{\ensuremath{\;\;\;\text{where}\;\;\;}}
\newcommand\suchthat{\ensuremath{\;\;\;\text{s.t.}\;\;\;}}
\newcommand\nn{\nonumber}
\newcommand\mcal[1]{\ensuremath{\mathcal{#1}}}
\newcommand\from[1]{\ensuremath{_{#1}}}
\renewcommand\to[1]{\ensuremath{^{#1}}}
\renewcommand\eqref[1]{Eq. (\ref{#1})}
\newcommand\pref[1]{(\ref{#1})}
\newcommand\upa{\ensuremath{\uparrow}}
\newcommand\dna{\ensuremath{\downarrow}}
\renewcommand\deg{\ensuremath{^\circ}}


%GREEK
\newcommand\bs[1]{\boldsymbol{#1}}
\newcommand\Om{\ensuremath{\Omega} }
\newcommand\om{\ensuremath{\omega}}
\newcommand\ep{\epsilon}
\newcommand\g{\gamma}
\newcommand\G{\Gamma}
\newcommand\D{\Delta}
\renewcommand\r{\rho}
\newcommand\m{\mu}
\newcommand\al{\alpha}
\newcommand\la{\ensuremath{\lambda} }


%FUNCTIONS
 \newcommand\csch{\text{csch}}
 \newcommand\inv[1]{\ensuremath{#1^{-1}}}
 \newcommand\vol[1]{\text{vol}(#1)} 
 \newcommand\ceiling[1]{\ensuremath{\left  \lceil #1 \right \rceil}}
 \newcommand\ceil[1]{\ensuremath{\left  \lceil #1 \right \rceil}}
 \newcommand\floor[1]{\ensuremath{ \left \lfloor #1 \right \rfloor}}
  \newcommand\sign[1]{\ensuremath{\text{sign}(#1)}}
 
 %LOGIC
 \newcommand\notimplies{\centernot\implies}

%%SET THEORY%%
\newcommand\seq{\ensuremath{\subseteq}}
\newcommand\nin{\ensuremath{\notin}}
\newcommand\union{\ensuremath{\cup}}
\newcommand\Union{\ensuremath{\bigcup}}
\newcommand\intersection{\ensuremath{\cap}}
\newcommand\intersect{\ensuremath{\cap}}
\newcommand\Intersection{\ensuremath{\bigcap}}
\newcommand\Intersect{\ensuremath{\bigcap}}
\newcommand\es{\ensuremath{\emptyset}}
 
%%ALGEBRA%%
\newcommand\leqc{\trianglelefteq}
 
 %%ANALYSIS
 \newcommand\hs{\ensuremath{\cal{H}} }
 \newcommand\lp[1]{\ensuremath{ {L^#1(\bb{R}^d)}}}
  \newcommand\LpRd{\ensuremath{ {L^P(\bb{R}^d)}}}
  \newcommand\Lp{\ensuremath{ {L^P}}}  
 \newcommand\Rd{\ensuremath{ {\bb{R}^d} }}
 \newcommand\R{\ensuremath{\bb{R}}}
  \newcommand\Q{\ensuremath{\bb{Q}} }
 \newcommand\Z{\ensuremath{\bb{Z}}}
 \newcommand\C{\ensuremath{\bb{C}}}
 \newcommand\N{\ensuremath{\bb{N}}}
 \newcommand\T{\ensuremath{\bb{T}}}
 \newcommand\Td{\ensuremath{ \bb{T}^d}}
 \newcommand\sa{$\sigma$\text{--algebra}}
 \newcommand\order[1]{\ensuremath{\cal{O}(#1)}}

%%PROBABILITY
 \newcommand\sigmafield{\ensuremath{\sigma\text{--field}}}
  \newcommand\lsystem{\ensuremath{\lambda\text{--system}}}
  \newcommand\psystem{\ensuremath{\pi \text{--system}}}
 
 %%PHYSICAL CONSTANTS
 \newcommand\h{\ensuremath{\hbar}}
 \newcommand\hb{\ensuremath{\hbar}}
 
 %%PHYSICS%%
 \newcommand\pb[2]{\ensuremath{ \left [#1, #2 \right ]_{PB}}} 

%% CALC/QUANTUM %%
\newcommand\abs[1]{\ensuremath{\left\vert #1\right \vert}}
\newcommand\Bigabs[1]{\ensuremath{\Bigl\vert #1 \Bigr\vert}}
\newcommand\biggabs[1]{\ensuremath{\biggl\vert #1 \biggr\vert}}

\newcommand\norm[1]{\ensuremath{\lvert \lvert  #1 \rvert \rvert }} 
\newcommand\Tr[1]{\ensuremath{\text{Tr}(#1) }} 
\newcommand\Bignorm[1]{\ensuremath{\Bigl\lvert \Bigl\lvert  #1 \Bigr\rvert \Bigr\rvert }} 

\newcommand\ip[1]{\ensuremath{\left\langle#1\right\rangle}}

\newcommand\avg[2][1]{
	\ifthenelse{\equal{#1}{1}}{\ensuremath{\left \langle#2\right \rangle }}{}
	\ifthenelse{\equal{#1}{2}}{\ensuremath{\left \langle#2^2\right \rangle }}{}
	\ifthenelse{\equal{#1}{3}}{\ensuremath{\left \langle#2\right \rangle ^2}}{}}

\newcommand\derv[3][1]{
	\ifthenelse{\equal{#1}{1}}{\ensuremath{\frac{d#2}{d#3}}}{ 
		\ensuremath{\frac{d^#1 #2}{d#3^#1}}
	}
}

\newcommand\totalderv[2]{\ensuremath{\frac{\text{d}#1}{\text{d}#2}}}

\newcommand\prtl[3][1]{	\ifthenelse{\equal{#1}{1}}{\ensuremath{\frac{\partial#2}{\partial#3}}}{}
	\ifthenelse{\equal{#1}{2}}{\ensuremath{\frac{\partial^2#2}{\partial#3^2}}}{}}
	
\newcommand\db{\ensuremath{\mathchar'26\mkern-12mu d}} 
	
\newcommand\bra[1]{\ensuremath{\langle#1\vert}}
 \newcommand\ket[1]{\ensuremath{\vert #1 \rangle}} 
 \newcommand\bk[2]{\ensuremath{\langle#1\vert#2\rangle}}
\newcommand\re[1]{\ensuremath{\text{Re}(#1)}}
\newcommand\im[1]{\ensuremath{\text{Im}(#1)}}

\newcommand\intall[2][x]{\int_{-\infty}^\infty #2 d #1}	


\newcommand\mathc{\ensuremath{\textbf{C}}}
\newcommand\mathr{\ensuremath{\textbf{R}}}
\newcommand\res{\ensuremath{\text{Res}}}
\newcommand\confeq{\ensuremath{\stackrel{\sim}{\rightarrow}}}	


%MATRIX%
\newcommand\qmatrix[9]{
 \left( \begin{array}{ccc}
 #1&#2&#3\\
 #4&#5&#6\\
 #7&#8&#9
 \end{array} \right)}
 
 \newcommand\sqmatrix[4]{
 \left( \begin{array}{cc}
 #1&#2\\
 #3&#4
  \end{array} \right)}

 \newcommand\matTWO[4]{
 \left( \begin{array}{cc}
 #1&#2\\
 #3&#4
  \end{array} \right)}

 \newcommand\matTHREE[9]{
 \left( \begin{array}{ccc}
 #1&#2&#3\\
 #4&#5&#6\\
 #7&#8&#9
  \end{array} \right)}
 

 \newcommand\mat[2][cccccccccccccccccccccc]{
 \left[ \begin{array}{#1} 
 #2\\
 \end{array} \right]}

% \newcommand\det[2][rrrrrrrrrrrrrrrrrrrrrrrrrrrrrr]{
 %\left| \begin{array}{#1} 
 %#2\\
 %\end{array} \right|}

 \newcommand\qdet[9]{
 \left\vert \begin{array}{ccc}
 #1&#2&#3\\
 #4&#5&#6\\
 #7&#8&#9
 \end{array} \right\vert}

  \newcommand\sqdet[4]{
 \left\vert \begin{array}{cc}
 #1&#2\\
 #3&#4
  \end{array} \right\vert}
 
 \newcommand\qvec[3]{
  \left( \begin{array}{c}
 #1\\
 #2\\
 #3
  \end{array} \right)}

 \newcommand\sqvec[2]{
  \left( \begin{array}{c}
 #1\\
 #2
  \end{array} \right)}
  
  %MATLAB%
  \newcommand\tril[1]{\ensuremath{\text{tril}(#1)}}
  \newcommand\diag[1]{\ensuremath{\text{diag}(#1)}}

  
%DISPLAY  

%ENVIRONMENTS

\theoremstyle{plain}   \newtheorem{thm}{Theorem}
\renewcommand{\qedsymbol}{$\blacksquare$}
\theoremstyle{plain}   \newtheorem{lem}{Lemma}
\theoremstyle{definition}   \newtheorem*{defn}{Definition}
\theoremstyle{plain}   \newtheorem{post}{Postulate}
\theoremstyle{plain}   \newtheorem{prop}{Proposition}
\theoremstyle{plain}   \newtheorem{cor}{Corollary} 
\newcommand\problem[1]{\textbf{Problem #1}\\}
%{\renewcommand{\descriptionlabel}[1]{}

\newcommand\pspace[2][20]{$\hspace{#1pt}$~}
\usepackage{graphicx}
\usepackage{pxfonts}
\usepackage{algorithm}
\usepackage{algorithmic}
\usepackage{epstopdf}
\usepackage{fullpage}
\usepackage{epic}
\usepackage{eepic}

\begin{document}

\title{CS5625 Final Project Proposal}
\author{Michael Flashman \\ mtf53@cornell.edu \and Tianhe Zhang \\ tz249@cornell.edu\\}
\date{\today}
\maketitle


\section{Overview}
%what we would like to do
%main chellenge generating physically based interactive model
%minor chellenge rendering (relief mapping)

\section{Game design}
% talk about the game

\section{Technical Components}
The main challenge for our game is how we create a real time physical interactive object such as our palm tree. We would like to experiment how the tree skeleton mesh interacts with different kinds of forces. In order to this, we are going to explore a third party physical engine. Other tasks includes how we can build the mesh for the tree model procedurally according to the skeleton mesh.\\

Another challenge is to make the tree look plausible. We would like to explore the hair optimization method described in [1] and use it to simulate the tree leaves and fronds. In this way, we make our tree model more efficient.\\

We also plan to implement other techniques if time allowed. For example, we can implement a normal/relief mapping for the bark on the trunk. We would also want to explore the wind model described in [2] so that we can see how the palm tree interacts with the wind.

%\begin{enumerate}
\section{Tentative Schedule}
\begin{itemize}
\item{Week 1: April 1 - April 7}

\begin{enumerate}
\item{Tree model: build a skeleton mesh procedurally for the palm tree rendering and physical simulation.}
\item{Explore physics engine and try to connect to our base code.}
\end{enumerate}

\item{Week 2: April 8 - April 14}

\begin{enumerate}
\item{Tree model: build the basic tree mesh procedurally by using the skeleton mesh.}
\item{Terrain: build a basic terrain using subdivision surface.}

\end{enumerate}

\item{Week 3: April 15 - April 21}

\begin{enumerate}
\item{Optimize tree simulation by using the interpolation of leaves and fronds based on the hair model.}

\end{enumerate}

\item{Week 4: April 22 - April 28}

\begin{enumerate}
\item{Apply the physics simulation to the skeleton mesh.}
\item{Apply certain techniques to make the tree look nice. e.g. Normal mapping for the trunck.}
\end{enumerate}

\item{Week 5: April 29 - May 5}

\begin{enumerate}
\item{Game implementation: Shooting a rock to the desert.}

\end{enumerate}

\item{Week 6: May 10 - May 15}

\begin{enumerate}
\item{Implement wind simulation similar to the method described in Chapter 6. GPU-Generated Procedural Wind Animations for Trees from Gem3. }
\end{enumerate}

\end{itemize}

\section{Citations}
\begin{thebibliography}{11}

\bibitem{1}
GPU gems 2 : Chapter 23. Hair Animation and Rendering in the Nalu Demo by Hubert Nguyen and William Donnelly

\bibitem{2}
GPU gems 3 : Chapter 6. GPU-Generated Procedural Wind Animations for Trees by Renaldas Zioma 


\bibitem{3}
GPU gem 3: Chapter 4. Next-Generation SpeedTree Rendering by Alexander Kharlamov, Iain Cantlay and Yury Stepanenko 

\bibitem{4}
GPU gem 3: Chapter 16. Vegetation Procedural Animation and Shading in Crysis by Tiago Sousa

\bibitem{5}
Real-time Terrain Rendering using Smooth Hardware Optimized Level of Detail by Bent Dalgaard, Larsen Niels and J�rgen Christensen

\bibitem{6}
Stochastic Dynamics: Simulating the Effects of Turbulence on Flexible Structures by Jos Stam

\end{thebibliography}
\end{document}
